\documentclass[../ap_2009.tex]{subfiles}

\graphicspath{{./image/}}

\begin{document}

\chapter{電磁気学: 電磁流体力学}
\section{}
(1)式に(2)-(5)式を代入して2次の微小項を無視していくと
\begin{equation}\begin{aligned}[b]
    m(n_0+\delta n)\pdv{\delta\vb*{u}}{t}
        +m(n_0+\delta n)\qty(\delta\vb*{u}\cdot\nabla)\delta\vb*{u}
        &= -q(n_0+\delta n)\qty(\delta\vb*{E}+\delta\vb*{u}\times\delta\vb*{B})\\
    mn_0\pdv{\delta\vb*{u}}{t}
        &= -qn_0\delta\vb*{E}\\
    m\pdv{\delta\vb*{u}}{t}
        &= -q\delta\vb*{E}
\end{aligned}\end{equation}

\section{}
(11)式をフーリエ変換すると
\begin{equation}\begin{aligned}[b]
    i\vb*{k}\times\delta\hat{\vb*{E}} &= i\omega\delta\hat{\vb*{B}}\\
    \delta\hat{\vb*{B}} &= \frac{1}{\omega}\vb*{k}\times\delta\hat{\vb*{E}}
\end{aligned}\end{equation}
これをフーリエ変換した(12)式に代入すると
\begin{equation}\begin{aligned}[b]
    i\vb*{k}\times\delta\vb*{B} &= -\mu_0qn_0\delta\vb*{u}-i\frac{\omega}{c^2}\delta\hat{\vb*{E}}\\
    \vb*{k}\times\qty(\frac{1}{\omega}\vb*{k}\times\delta\hat{\vb*{E}})
        &= -i\mu_0qn_0\delta\vb*{u}+\frac{\omega}{c^2}\delta\hat{\vb*{E}}\\
    \vb*{k}\times\qty(\vb*{k}\times\delta\hat{\vb*{E}})
        &= -i\mu_0qn_0\omega\delta\vb*{u}+\frac{\omega^2}{c^2}\delta\hat{\vb*{E}}\\
\end{aligned}\end{equation}

\section{}
(6)をフーリエ変換すると
\begin{equation}\begin{aligned}[b]
    -im\omega\delta\hat{\vb*{u}}&=-q\delta\hat{\vb*{E}}\\
    \delta\hat{\vb*{u}} &= -i\frac{q}{m\omega}\delta\hat{\vb*{E}}
\end{aligned}\end{equation}
(13)式を整理していく。左辺はベクトル解析の公式より
\begin{equation}\begin{aligned}[b]
    \vb*{k}\times\qty(\vb*{k}\times\delta\hat{\vb*{E}})
        &= (\vb*{k}\cdot\delta\hat{\vb*{E}})\vb*{k}-k^2\delta\hat{\vb*{E}}
        = -k^2\delta\hat{\vb*{E}}
\end{aligned}\end{equation}
なので(13)式は
\begin{equation}\begin{aligned}[b]
    k^2\delta\hat{\vb*{E}} &= -\mu_0\epsilon_0\frac{q^2n_0}{\epsilon_0 m}\delta\hat{\vb*{E}}+\frac{\omega^2}{c^2}\delta\hat{\vb*{E}}\\
    0 &= \qty(\omega^2-k^2c^2-\frac{q^2n_0}{\epsilon_0m})\delta\hat{\vb*{E}}
\end{aligned}\end{equation}
電場の振動成分がある、つまり\(\delta\hat{\vb*{E}}\neq0\)なので
\begin{equation}\begin{aligned}[b]
    0 &=\omega^2-k^2c^2-\frac{q^2n_0}{\epsilon_0m} \\
    \omega^2 &=k^2c^2+\frac{q^2n_0}{\epsilon_0m}
\end{aligned}\end{equation}

\section{}
分散関係が成り立ち、電場の振動成分があるので(11)式のフーリエ変換の式を見ると
\begin{equation}\begin{aligned}[b]
    \delta\hat{\vb*{B}} &= \frac{1}{\omega}\vb*{k}\times\delta\hat{\vb*{E}} \neq 0
\end{aligned}\end{equation}

\section{}
(14)式をプラズマ振動数を用いて表すと
\begin{equation}\begin{aligned}[b]
    \omega =\sqrt{k^2c^2+\omega_p^2}
\end{aligned}\end{equation}
なので位相速度は
\begin{equation}\begin{aligned}[b]
    v_p = \frac{\omega}{k} = \sqrt{c^2+\frac{\omega_p^2}{k^2}} > c
\end{aligned}\end{equation}
群速度は
\begin{equation}\begin{aligned}[b]
    v_g = \pdv{\omega}{k} = \frac{c^2k}{\sqrt{k^2c^2+\omega_p^2}} = \frac{c}{\sqrt{1+(\omega_p/ck)^2}}<c
\end{aligned}\end{equation}
より
\begin{equation}\begin{aligned}[b]
    v_g < c < v_p
\end{aligned}\end{equation}

\section{}
屈折率\(N\)を\(\omega\)の関数で表すと
\begin{equation}\begin{aligned}[b]
    N = \frac{c}{v_p} = \frac{ck}{\omega}=\sqrt{1-\frac{\omega_p^2}{\omega^2}}
\end{aligned}\end{equation}
これより振動数\(\omega<\omega_p\)の光は金属中の電子ガス中には存在できないため、
金属に入射した光は全て反射する。これが金属光沢の由来である。

\section*{感想}
誘導に乗っていくだけではあるが、
プラズマ振動数の名前の由来がそのまんま金属中のプラズマの振動数と知れたのは面白かった。



\end{document}
