\documentclass[../ap_2010.tex]{subfiles}

\graphicspath{{./image/}}

\begin{document}

\chapter{電磁気学: 共振器}
\section{}
光学長は\(nL\)であるので位相変化は
\begin{equation}\begin{aligned}[b]
    \theta = \frac{\omega}{c}nL = \frac{\omega nL}{c}
\end{aligned}\end{equation}
\section{}
\begin{equation}\begin{aligned}[b]
    2\pi m &= \frac{\omega_m nL}{c}\\
    \omega_m &= \frac{2\pi c}{nL} m
\end{aligned}\end{equation}

\section{}
\begin{equation}\begin{aligned}[b]
    E(t) &= \frac{E_0}{2}\qty[\cos(\frac{2\pi ct}{nL}(m-1))+2\cos(\frac{2\pi ct}{nL}m)+\cos(\frac{2\pi ct}{nL}(m+1))]\\
    &= E_0\qty(1+\cos\frac{2\pi ct}{nL})\cos\frac{2\pi ct}{nL}m\\
    &= 2E_0\cos^2\frac{\pi ct}{nL} \cos\frac{2\pi ct}{nL}m
\end{aligned}\end{equation}
いま\(n=1\)の時を考えているので、包絡線は
\begin{equation}\begin{aligned}[b]
    A(t) = 2E_0\cos^2\frac{\pi ct}{L}
\end{aligned}\end{equation}
で極大となる時間間隔は\(L/c\)である。

\section{}
屈折率は角振動数と波数を用いて
\begin{equation}\begin{aligned}[b]
    n = \frac{ck}{\omega} = \frac{c}{\omega}\qty(k_0+\frac{\omega}{v_g}) = \frac{ck_0}{\omega}+\frac{c}{v_g}
\end{aligned}\end{equation}
と表せる。

\section{}
波数が変わったので、これをもとに[1]と[2]で行ったのと同様のことをすると
\begin{equation}\begin{aligned}[b]
    2\pi m &= kL = k_0L+\frac{\omega L}{v_g}\\
    \omega &= -v_gk_0+\frac{2\pi v_g}{L}m
\end{aligned}\end{equation}

\section{}
設問[3]と同様にやって
\begin{equation}\begin{aligned}[b]
    E(t) &= \frac{E_0}{2}\qty(\cos\qty[\qty(-v_gk_0t+\frac{2\pi v_gt}{L})-\frac{2\pi v_gt}{L}]
    +\cos\qty[-v_gk_0t+\frac{2\pi v_gt}{L}]
    +\cos\qty[\qty(-v_gk_0t+\frac{2\pi v_gt}{L})+\frac{2\pi v_gt}{L}])\\
    &= E_0\qty(1+\cos\frac{2\pi v_g}{L}t)\cos\qty[-v_gk_0t+\frac{2\pi v_gt}{L}]\\
    &= 2E_0\cos^2\frac{\pi v_gt}{L}\cos\qty[-v_gk_0t+\frac{2\pi v_gt}{L}]
\end{aligned}\end{equation}
これより包絡線の隣り合う極大間の時間間隔は\(L/v_g\)

\section{}
\(i\)番目の極大時刻は\(t=iL/v_g\)であるので、\(\Delta\phi\)は
\begin{equation}\begin{aligned}[b]
    \Delta\phi &= \abs{\qty(-v_gk_0+\frac{2\pi v_g}{L}m)\frac{L}{v_g}} = k_0L = \abs{\frac{\omega_0}{\omega_{rep}}}
\end{aligned}\end{equation}
分散がないときには\(k=\omega/c\)より、分散があるときの式を\(k_0=0,v_g=c\)にすることで得られる。
\(\Delta\phi\)は\(k_0\)によって生じているのがわかるので、
これより光に分散関係がある、つまり群速度と位相測度が違うことによって\(\Delta\phi\)が生じているのがわかる。

\section*{感想}
設問3で隣接する共振モード3本を外部に取り出すとあるが、
実際どうやって\(1:2:1\)の振幅になるように取り出しているのだろうか?



\end{document}
