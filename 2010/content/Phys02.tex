\documentclass[../ap_2010.tex]{subfiles}

\graphicspath{{./image/}}

\begin{document}

\chapter{統計力学: 固体比熱}
\section{}
フェルミ分布を描く(略)。
\section{}
電子数\(N\)は状態密度と分布関数を用いて表して、ゾンマーフェルト展開を通して式を整理していく。
\begin{equation}\begin{aligned}[b]
    N &= \int_{-\infty}^{\infty}f(\varepsilon)D(\varepsilon)d\varepsilon\\
    &= \int_{-\infty}^{\mu}D(\varepsilon)d\varepsilon + \frac{\pi^2}{6}(k_BT)^2\dv{D(\mu)}{\varepsilon}+\mathcal{O}(T^4)\\
    &= \int_{0}^{\varepsilon_F}D(\varepsilon)d\varepsilon
    +\int_{\varepsilon_F}^{\mu}D(\varepsilon)d\varepsilon
    + \frac{\pi^2}{6}(k_BT)^2\dv{D(\varepsilon_F)}{\varepsilon}+\mathcal{O}(T^4)\\
    &= N  +(\mu-\varepsilon_F)D_0+ \frac{\pi^2}{6}(k_BT)^2\dv{D(\varepsilon_F)}{\varepsilon}+\mathcal{O}(T^4)
\end{aligned}\end{equation}
よって
\begin{equation}\begin{aligned}[b]
    \mu = \varepsilon_F- \frac{\pi^2}{6D_0}\dv{D(\varepsilon_F)}{\varepsilon}\,(k_BT)^2
\end{aligned}\end{equation}

\section{}
\begin{equation}\begin{aligned}[b]
    E(T) &= \int_{-\infty}^{\infty}f(\varepsilon)\varepsilon D(\varepsilon)d\varepsilon\\
    &\simeq \int_{0}^{\varepsilon_F}\varepsilon D(\varepsilon)d\varepsilon
    +\int_{\varepsilon_F}^{\mu}\varepsilon D(\varepsilon)d\varepsilon
    +\left.\frac{\pi^2}{6}(k_BT)^2\dv{\varepsilon}\qty(\varepsilon D(\varepsilon))\right|_{\varepsilon=\varepsilon_F}\\
    &\simeq E(0)+\varepsilon_FD_0(\mu-\varepsilon_F)+\frac{\pi^2\varepsilon_F}{6}(k_BT)^2\dv{D(\varepsilon_F)}{\varepsilon}+\frac{\pi^2}{6}(k_BT)^2D_0\\
    &= E_0 + \frac{\pi^2}{6}D_0(k_BT)^2\\
    C(T) &= \dv{E}{T}=\frac{\pi^2}{3}D_0k_B^2T
\end{aligned}\end{equation}

\section{}
ボーズ分布を描く(略)。

\section{}
見慣れないnotationだが、イバッハ・リュートでの求め方はこんな感じ。
\begin{equation}\begin{aligned}[b]
    \rho(\omega) = \frac{V}{(2\pi)^3}\int_\omega \frac{dk}{\abs{\grad_k \omega(k)}}
    = \frac{V}{(2\pi)^3}\frac{4\pi k(\omega)^2}{v} = \frac{V}{2\pi^2 v^3}\omega^2
\end{aligned}\end{equation}
これで求めたのは状態密度であって状態数ではないかも。
\section{}
フォノンの振動数の上限を\(\omega_D\)とおく。
このときフォノンの内部エネルギーは低温であることを使いながら変形すると
\begin{equation}\begin{aligned}[b]
    E &= \int_0^{\omega_D}\frac{3\rho(\omega)\hbar\omega}{e^{\beta\hbar\omega}-1}d\omega\\
    &= \frac{3V(k_BT)^4}{2\pi^2\hbar^3v^3}\int_{0}^{\infty}\frac{(\beta\hbar\omega)^3}{e^{\beta\hbar\omega}-1}d(\beta\hbar\omega)\\
    &= \frac{\pi^2Vk_B^4}{10\hbar^3v^3}T^4\\
    C &= \dv{E}{T}=\frac{2\pi^2Vk_B^4}{5\hbar^3v^3}T^3
\end{aligned}\end{equation}

\section{}
電子比熱とフォノン比熱をそれぞれ\(\alpha T,\gamma T^3\)のように表すと
\begin{equation}\begin{aligned}[b]
    C_V &= \alpha T+\gamma T^3\\
    \frac{C_V}{T} &= \alpha +\gamma T
\end{aligned}\end{equation}
となる。
これより\(C_V/T\)を\(T^2\)にたいしてプロットすると直線状に並ぶことがわかる。
\(T\to 0\)の外挿で電子比熱の\(\alpha\)が、傾きからフォノン比熱の\(\gamma\)が求まる。
また、\(C_T/T\)が\(T\to 0\)で\(0\)になるのは比熱に電子比熱が関わっていないことを表す。
比熱はフェルミ面付近の電子が担っているので、
このような結晶は半導体や絶縁体である。

\section*{感想}
固体の比熱の話の総おさらい。
初見のときにゾンマーフェルト展開を\(\varepsilon_F\)でいったん発想が思いつかないけど、
物性をいろいろやってると結構自然な発想だというのに気付けた。

低温での半導体の比熱の測定とかやったことないけど、
本当にそうなるのか?

\end{document}
