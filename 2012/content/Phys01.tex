\documentclass[../ap_2012.tex]{subfiles}

\graphicspath{{./image/}}

\begin{document}

\setcounter{chapter}{0}
\chapter{古典力学:拘束力・回転運動}
\section{}
質点の運動を拘束する張力\(T\)は運動の方向に垂直なので
\subsection*{(i)}
\(dW=T\dot vdt=0\)より仕事をしないから、運動エネルギーは保存する。
\subsection*{(ii)}
\(dN=T\times vdt\neq 0\)よりモーメントが生じるから、角運動量は保存しない。

\section{}
点Qの座標は\((a\cos\varphi,a\sin\varphi)\)で、
糸の長さは\(l_0-a\varphi\)より点Qからみた質点の位置は
\(-(l_0-a\varphi)\sin\varphi,(l_0-a\varphi)\cos\varphi\)なので、
質点の座標は
\begin{equation}\begin{aligned}[b]
    r(t) = \Bigl(a\cos\varphi-(l_0-a\varphi)\sin\varphi,a\sin\varphi+(l_0-a\varphi)\cos\varphi\Bigr)
\end{aligned}\end{equation}
となる。

これより速度ベクトルは
\begin{equation}\begin{aligned}[b]
    \dot{r}(t) =-(l_0-a\varphi)\dot{\varphi}\Bigl(\cos\varphi,\sin\varphi\Bigr)
\end{aligned}\end{equation}

角運動量は
\begin{equation}\begin{aligned}[b]
    L&=-m(l_0-a\varphi)\dot{\varphi}\qty[
        \qty{a\cos\varphi-(l_0-a\varphi)\sin\varphi}\sin\varphi
        -\qty{a\sin\varphi+(l_0-a\varphi)\cos\varphi}\cos\varphi]\\
    &=m(l_0-a\varphi)^2\dot{\varphi}
\end{aligned}\end{equation}

\section{}
\begin{equation}\begin{aligned}[b]
    \abs{\dot{r}}=\abs{(l_0-a\varphi)\dot{\phi}} = \abs{\dot{r}(0)}=v_0
\end{aligned}\end{equation}
より
\begin{equation}\begin{aligned}[b]
    (l_0-a\varphi)\dot{\varphi} = v_0
\end{aligned}\end{equation}
両辺を時間で積分して
\begin{equation}\begin{aligned}[b]
    l_0-\frac{a}{2}\varphi^2 = v_0t
\end{aligned}\end{equation}
\(\varphi=l/a\)となるときが求める時間\(\tau\)なので
\begin{equation}\begin{aligned}[b]
    \frac{l_0^2}{a}-\frac{l_0^2}{2a}&=v_0\tau\\
    \tau &= \frac{2a v_0}{l_0^2}
\end{aligned}\end{equation}

\section{}
\begin{equation}\begin{aligned}[b]
    T = \frac{mv_0^2}{l_0-a\varphi}\Bigl(\sin\varphi, -\cos\varphi\Bigr)
\end{aligned}\end{equation}
これは、微小時間間隔でみると、半径\(l_0-a\varphi\)の等速円運動となる。

\section{}
張力のモーメントは
\begin{equation}\begin{aligned}[b]
    N &= \frac{mv_0^2}{l_0-a\varphi}\qty[
        -\qty{a\cos\varphi-(l_0-a\varphi)\sin\varphi}\cos\varphi
        -\qty{a\sin\varphi+(l_0-a\varphi)\cos\varphi}\sin\varphi]\\
    &=-\frac{mv_0^2}{l_0-a\varphi}a = -mv_0 a\dot{\varphi}
\end{aligned}\end{equation}
角運動量の時間微分は
\begin{equation}\begin{aligned}[b]
    \dv{L}{t} &= \dv{t}\qty(m(l_0-a\varphi)^2\dot{\varphi})\\
    &=mv_0\dv{t}\qty(l_0-a\varphi) = -mv_0 a\dot{\varphi}
\end{aligned}\end{equation}
となり確かに
\begin{equation}\begin{aligned}[b]
    \dv{t}L(t)=N(t)
\end{aligned}\end{equation}
が成り立っているのがわかる。

\section*{感想}
設問[4]は先に結果がわかったので、それを満たすように張力をいきなり出してごまかしたけど、
正攻法はどうすべきだったのだろうか?

\end{document}
