\documentclass[../ap_2012.tex]{subfiles}

\graphicspath{{./image/}}

\begin{document}

\setcounter{chapter}{0}
\chapter{量子力学:角運動量・p電子系}
\section{}
\begin{equation}\begin{aligned}[b]
    J_x &= \begin{pmatrix}
        \mel{1,+1}{J_x}{1,+1} & \mel{1,+1}{J_x}{1,0} & \mel{1,+1}{J_x}{1,-1}\\
        \mel{1,0}{J_x}{1,+1} & \mel{1,0}{J_x}{1,0} & \mel{1,0}{J_x}{1,-1}\\
        \mel{1,-1}{J_x}{1,+1} & \mel{1,-1}{J_x}{1,0} & \mel{1,-1}{J_x}{1,-1}
    \end{pmatrix}\\
    &= \frac{1}{2}\begin{pmatrix}
        \mel{1,+1}{(J_++J_-)}{1,+1} & \mel{1,+1}{(J_++J_-)}{1,0} & \mel{1,+1}{(J_++J_-)}{1,-1}\\
        \mel{1,0}{(J_++J_-)}{1,+1} & \mel{1,0}{(J_++J_-)}{1,0} & \mel{1,0}{(J_++J_-)}{1,-1}\\
        \mel{1,-1}{(J_++J_-)}{1,+1} & \mel{1,-1}{(J_++J_-)}{1,0} & \mel{1,+-}{(J_++J_-)}{1,-1}
    \end{pmatrix}\\
    &= \frac{\hbar}{\sqrt{2}}\begin{pmatrix}
        0 & 1 & 0\\
        1 & 0 & 1\\
        0 & 1 & 0
    \end{pmatrix}
\end{aligned}\end{equation}
\begin{equation}\begin{aligned}[b]
    J_y &= \begin{pmatrix}
        \mel{1,+1}{J_y}{1,+1} & \mel{1,+1}{J_y}{1,0} & \mel{1,+1}{J_y}{1,-1}\\
        \mel{1,0}{J_y}{1,+1} & \mel{1,0}{J_y}{1,0} & \mel{1,0}{J_y}{1,-1}\\
        \mel{1,-1}{J_y}{1,+1} & \mel{1,-1}{J_y}{1,0} & \mel{1,-1}{J_y}{1,-1}
    \end{pmatrix}\\
    &= \frac{1}{2i}\begin{pmatrix}
        \mel{1,+1}{(J_+-J_-)}{1,+1} & \mel{1,+1}{(J_+-J_-)}{1,0} & \mel{1,+1}{(J_+-J_-)}{1,-1}\\
        \mel{1,0}{(J_+-J_-)}{1,+1} & \mel{1,0}{(J_+-J_-)}{1,0} & \mel{1,0}{(J_+-J_-)}{1,-1}\\
        \mel{1,-1}{(J_+-J_-)}{1,+1} & \mel{1,-1}{(J_+-J_-)}{1,0} & \mel{1,+-}{(J_+-J_-)}{1,-1}
    \end{pmatrix}\\
    &= \frac{\hbar}{\sqrt{2}}\begin{pmatrix}
        0 & -i & 0\\
        i & 0 & -i\\
        0 & i & 0
    \end{pmatrix}
\end{aligned}\end{equation}
\begin{equation}\begin{aligned}[b]
    J_z &= \hbar\begin{pmatrix}
        1 & 0 & 0\\
        0 & 0 & 0\\
        0 & 0 & -1
    \end{pmatrix}
\end{aligned}\end{equation}

\section{}
ハミルトニアンは
\begin{equation}\begin{aligned}[b]
    H_M = -\gamma B J_z
\end{aligned}\end{equation}
であり、\(z\)軸を対角化の基底に取っているので固有値は
\begin{equation}\begin{aligned}[b]
    E= -\gamma \hbar B,\quad 0,\quad\gamma\hbar B
\end{aligned}\end{equation}
となる。

\section{}
相互作用表示したときの状態ベクトルの時間発展の式より
\begin{equation}\begin{aligned}[b]
    i\hbar\pdv{t}\sum_m c_m(t) \ket{1,m}&=-2\gamma B_{RF}J_x\sum_m c_m(t) \ket{1,m}\\
    i\hbar\pdv{t}\begin{pmatrix}
        {c_1(t)}\\
        {c_0(t)}\\
        {c_{-1}(t)}
    \end{pmatrix}
    &=-2\gamma \hbar B_{RF}\frac{1}{\sqrt{2}}\begin{pmatrix}
        0 & 1 & 0\\
        1 & 0 & 1\\
        0 & 1 & 0
    \end{pmatrix}
    \begin{pmatrix}
        {c_1(t)}\\
        {c_0(t)}\\
        {c_{-1}(t)}
    \end{pmatrix}
\end{aligned}\end{equation}
\(J_x\)は
\begin{equation}\begin{aligned}[b]
    \begin{pmatrix}
        1 & 0 & 0\\
        0 & 0 & 0\\
        0 & 0 & -1
    \end{pmatrix}
    =\begin{pmatrix}
        1 & 1/\sqrt{2} & 1\\
        1/\sqrt{2} & 0 & -1/\sqrt{2}\\
        1 & -1/\sqrt{2} & 1\\
    \end{pmatrix}
    J_x\begin{pmatrix}
        1 & 1/\sqrt{2} & 1\\
        1/\sqrt{2} & 0 & -1/\sqrt{2}\\
        1 & -1/\sqrt{2} & 1
    \end{pmatrix}
\end{aligned}\end{equation}
によって対角化できるので、

\section*{感想}




\end{document}
