\documentclass[../../master.tex]{subfiles}

\graphicspath{{./image/}}

\begin{document}

\chapter{統計力学: 合金}
\section{}
1格子につき、元素間の結合は6本ある。
基底状態ではそれらすべてがX, Y間の結合であるので、
基底状態のエネルギー\(E_0\)は
\begin{equation}\begin{aligned}[b]
    E_0 = 6q
\end{aligned}\end{equation}

\section{}
格子点\(i\)と\(j\)にある元素が同じときのエネルギーが\(p\),
格子点\(i\)と\(j\)にある元素が違うときのエネルギーが\(q\)となるようように
\(S_i,\,S_j\)を使って書くと
\begin{equation}\begin{aligned}[b]
    \frac{1+S_iS_j}{2}p + \frac{1-S_iS_j}{2}q = \frac{p-q}{2}S_iS_j + \frac{p+q}{2}
\end{aligned}\end{equation}
これをすべての\(\ev{i,j}\)で和を取ったのがハミルトニアンなので
\begin{equation}\begin{aligned}[b]
    H = \sum_{\ev{i.j}}\qty(\frac{p-q}{2}S_iS_j + \frac{p+q}{2})
    \equiv \sum_{\ev{i.j}}\qty(JS_iS_j + K)
\end{aligned}\end{equation}
ここで
\begin{equation}\begin{aligned}[b]
    J := \frac{p-q}{2},\qquad  K := \frac{p+q}{2}
\end{aligned}\end{equation}

\section{}
ハミルトニアンに\(S_i = \ev{S_i}+\delta S_i\)を代入して、
和をとって整理していく。
\begin{equation}\begin{aligned}[b]
    H &= \sum_{\ev{i,j}}\Bigl[J\qty(\ev{S_i}+\delta S_i)(\ev{S_i}+\delta S_i)+K\Bigr]
    = \sum_{\ev{i,j}}\Bigl[-n^2J+nJ\delta S_i-nJ\delta S_k +J\delta S_i \delta S_j+K\Bigr]\\
    &\simeq 6nJ\qty(\sum_i \delta S_i - \sum_j \delta S_j)-3Nn^2J+3NK
    = 6nJ\qty(\sum_i S_i - \sum_j S_j)+3N\qty(n^2J+K)
\end{aligned}\end{equation}

\section{}
すべての\(S_i\)に\(\pm 1\)がどのように入っているかの配列を\(\{S_i\}\)のように表すとする。
分配関数は
\begin{equation}\begin{aligned}[b]
    Z &= \sum_{\{S_i\}}\sum_{\{S_j\}}\exp[-6\beta nJ\qty(\sum_i S_i - \sum_j S_j)-3\beta N\qty(n^2J+K)]\\
    &= \exp[-3\beta N\qty(n^2J+K)]\sum_{\{S_i\}}\sum_{\{S_j\}}\prod_{ij}\exp[-6\beta nJ\qty(S_i - S_j)]\\
    &= \exp[-3\beta N\qty(n^2J+K)]\prod_{ij} \sum_{S_i = \pm 1}\sum_{S_j = \pm 1}\exp[-6\beta nJ\qty(S_i - S_j)]\\
    &= \exp[-3\beta N\qty(n^2J+K)]\Bigl[2(1+\cosh(12\beta nJ))\Bigr]^{N/2}\\
    &= \Bigl[2\exp[-3\beta(n^2J+K)]\cosh(6\beta nJ)\Bigr]^N
    =  \qty[2\exp[-3\frac{n^2J+K}{k_BT}]\cosh(\frac{6nJ}{k_BT})]^N
\end{aligned}\end{equation}

\section{}
\begin{equation}\begin{aligned}[b]
    \pdv{J}\ln Z
    &= \frac{1}{Z}\sum_{\{S_i\}}\sum_{\{S_j\}}\pdv{J} \exp[-6\beta nJ\qty(\sum_i S_i - \sum_j S_j)-3\beta N\qty(n^2J+K)]\\
    &= \frac{1}{Z}\sum_{\{S_i\}}\sum_{\{S_j\}}\qty{-6\beta n\qty(\sum_i S_i - \sum_j S_j)-3\beta Nn^2} \exp[-6\beta nJ\qty(\sum_i S_i - \sum_j S_j)-3\beta N\qty(n^2J+K)]\\
    &= \ev{-6\beta n\qty(\sum_i S_i - \sum_j S_j)-3\beta Nn^2}
    = -6\beta n\qty(\sum_i \ev{S_i} - \sum_j \ev{S_j})-3\beta Nn^2\\
    &= -3\beta nN\qty(n_A-n_B+n)
\end{aligned}\end{equation}
これと\(n_A+n_B = 0\)を使うと\(N_A\)は
\begin{equation}\begin{aligned}[b]
    n_A &= -\frac{1}{6\beta nN}\pdv{J}\ln Z -\frac{n}{2}\\
    &= -\frac{1}{6\beta nN}\pdv{J}\ln[2\exp[-3\frac{n^2J+K}{k_BT}]\cosh(\frac{6nJ}{k_BT})]^N - \frac{n}{2}\\
    &= -\frac{1}{6\beta n}\Bigl[-3\beta n^2+6n\tanh(6\beta n J)\Bigr] - \frac{n}{2}\\
    &= -\tanh(6\beta nJ) = - \tanh(\frac{6nJ}{k_BT})
\end{aligned}\end{equation}
そして\(n_B\)は
\begin{equation}\begin{aligned}[b]
    n_B = -n_A = \tanh(\frac{6nJ}{k_BT})
\end{aligned}\end{equation}

\section{}
\(n_A = -n_B = -n\)より
\begin{equation}\begin{aligned}[b]
    n = \tanh(6\beta nJ)
\end{aligned}\end{equation}

\section{}
自己無同撞着方程式で\(n\neq0\)となる解がある条件は\(\tanh\)のグラフを考えると、
\(6J/k_BT>1\)のときであるので臨界温度\(T_c\)は
\begin{equation}\begin{aligned}[b]
    T_c = \frac{6J}{k_B}.
\end{aligned}\end{equation}

また、\(T = T_c-\delta T\)の\(T_c\)近傍の定温相では\(n\ll 1\)であるので\(\tanh\)を展開して
\begin{equation}\begin{aligned}[b]
    n &= \frac{6nJ}{k_B(T_c-\delta T)} -\frac{1}{3}\qty(\frac{6\beta nJ}{k_B(T_c-\delta T)})^3\\
    &\simeq n\frac{6J}{k_BT_c}\qty(1+\frac{\delta T}{T_c})-\frac{n^3}{3}\qty(\frac{6J}{k_BT_c})^3\qty(1+3\frac{\delta T}{T_c})
    \simeq n\qty(1+\frac{\delta T}{T_c})-\frac{n^3}{3}\\
    0 &= n\qty(\frac{\delta T}{T_c}-\frac{n^2}{3})\\
    n &= 0, \pm\sqrt{\frac{3\delta T}{T_c}}
\end{aligned}\end{equation}
となる。

これらの式をまとめると、高温相では\(n\)は常に\(0\)で、
臨界温度\(T_c\)を超えると臨界温度を規準とした温度の\(1/2\)乗に比例して\(n\)が大きくなっていく。

\section{}
高温相では各副サイトにX,Yの元素不規則に配置している。
合金を冷やしていき、設問[6]で求めた臨界温度\(T_c\)より低くなると、
徐々に秩序だった配置に元素が入っていく。
そして十分低温になると、
自己無撞着方程式より\(n=\pm 1\)になることから基底状態である塩化ナトリウム型の結晶構造になることがわかる。


\section*{感想}
『説明を理解しながら以下の設問[1]-[8]に答えよ。』とあるので、
その場で解いていくタイプの問題ではあるが驚いた。
おそらくキッテルの下巻に乗ってたり、
結晶成長の実験をやる人だったりと、
やったことある人はいるのかもしれない。

いわゆる最適化アルゴリズムの焼きなまし法の元ネタで、
高温の無秩序相から定温の秩序相にすると基底状態が得られるというやつ。
実際の試料・アルゴリズムともに1回の冷却だと局所安定状態が得られるので、
何回か加熱して冷やすというのを繰り返すのも同じである。

合金系は適切に変数変換をすると反強磁性のイジング模型になるのは、
相転移周りの具体例を増やす上でも良い問題だった。
他にも気液相転移も分子の密度を秩序変数を変換すると強磁性のイジング模型になったはず。



\end{document}
