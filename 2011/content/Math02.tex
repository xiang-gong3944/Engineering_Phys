\documentclass[../ap_2011.tex]{subfiles}

\graphicspath{{./image/}}

\begin{document}

\setcounter{chapter}{1}
\chapter{線形代数: 対称行列と2次形式}
\section{}
アインシュタインの縮約を使い、ベクトル表示と成分表示を行き来することにする。
\(x^\mathsf{T}Ax/2-x^\mathsf{T}b\)の停留値となる\(x\)の条件は
\begin{equation}\begin{aligned}[b]
    0&=\nabla\qty( \frac{1}{2}x^\mathsf{T}Ax-x^\mathsf{T}b) = \pdv{x_i}\qty(\frac{1}{2}A_{jk}x_jx_k-x_jb_j)\\
    &=\frac{1}{2}\delta_{ij}A_{jk}x_k+\frac{1}{2}\delta_{ik}A_{jk}x_j-\delta_{ij}b_j=\frac{A_{ij}+A_{ji}}{2}x_j-b_i
\end{aligned}\end{equation}
いま、\(A\)は対称行列で\(A_{ij}=A_{ji}\)より
\begin{equation}\begin{aligned}[b]
    0&=A_{ij}x_j-b=Ax-b\\
    Ax &= b
\end{aligned}\end{equation}
これを逆にたどれば、\(Ax=b\)を\(x\)について解いたものは\(x^\mathsf{T}Ax/2-x^\mathsf{T}b\)の停留値となるのもわかるので示せた。

\section{}
\(B\)の固有値方程式を考える。
固有値を\(\lambda\)とすると、
\begin{equation}\begin{aligned}[b]
    0 &= \abs{B-cI}
    =\begin{vmatrix}
        6-\lambda & 2 & 2\\
        2 & 7-\lambda & 0\\
        2 & 0 & 5-\lambda
    \end{vmatrix}
    =(5-\lambda)\begin{vmatrix}
        6-\lambda & 2 \\
        2 & 7-\lambda
    \end{vmatrix}
    +2\begin{vmatrix}
        2 & 2\\
        7-\lambda & 0
    \end{vmatrix}\\
    &=(5-\lambda)(6-\lambda)(7-\lambda)-4(5-\lambda)-4(7-\lambda)
    =(5-\lambda)(6-\lambda)(7-\lambda)-8(6-\lambda)\\
    &=(6-\lambda)(\lambda^2-12\lambda+27)=(6-\lambda)(3-\lambda)(9-\lambda)
\end{aligned}\end{equation}
これより\(c_1=9,c_2=6,c_3=3\)となる。
また、\(A\)は対称行列であるため、
固有値に対応する大きさ1の固有ベクトルを並べることで\(A\)を対角化する\(D\)を作ることができる。
よって\(D\)は
\begin{equation}\begin{aligned}[b]
    D =\begin{pmatrix}
        2/3&1/3&2/3\\
        2/3&-2/3&-1/3\\
        1/3&2/3&-2/3
    \end{pmatrix}
\end{aligned}\end{equation}

\section{}
\begin{equation}\begin{aligned}[b]
    B^k\begin{pmatrix}
        2 \\1\\0
    \end{pmatrix}
    =B^k\qty(2\begin{pmatrix}
        2/3\\2/3\\1/3
    \end{pmatrix}
    +\begin{pmatrix}
        2/3\\-1/3\\-2/3
    \end{pmatrix})
    =\qty(2\times9^k\begin{pmatrix}
        2/3\\2/3\\1/3
    \end{pmatrix}
    +3^k\begin{pmatrix}
        2/3\\-1/3\\-2/3
    \end{pmatrix})
\end{aligned}\end{equation}
これより
\begin{equation}\begin{aligned}[b]
    \frac{a_k}{b_k}=\frac{4\times 9^k+2\times 3^k}{4\times 9^k- 3^k}\to 1\quad(k\to\infty)
\end{aligned}\end{equation}
となる。

\section{}
\(y=(y_1\,y_2\,y_3)^\mathsf{T}, z=(z_1\,z_2\,z_3)^\mathsf{T}\)とする。
このとき、\(D\)は直交行列であるので\(z^\mathsf{T}z=y^\mathsf{T}y\)が成り立つことに注意して
\(f\)を変形していく。
\begin{equation}\begin{aligned}[b]
    f
    =\frac{1}{4} +\frac{y^\mathsf{T} B y^\mathsf{T}}{4y^\mathsf{T}y}
    =\frac{1}{4} +\frac{z^\mathsf{T} C z^\mathsf{T}}{4z^\mathsf{T}z}
    =\frac{1}{4} +\frac{9z_1^2+6z_2^2+3z_3^2}{4z^\mathsf{T}z}
    = 1 +\frac{3}{4}\frac{3z_1^2+2z_2^2}{z^\mathsf{T}z}
\end{aligned}\end{equation}
ここで、\(z_1=r\sin\theta\cos\varphi,z_2=r\sin\theta\sin\varphi,z_3=r\cos\theta\)として極座標で表示すると
\begin{equation}\begin{aligned}[b]
    f=1+\frac{3}{4}\sin^2\theta(3\cos^2\varphi+2\sin^2\varphi)
    =1+\frac{3}{4}\sin^2\theta(\cos^2\varphi+2)\geq 1
\end{aligned}\end{equation}
となるので最小値は 1

\end{document}
