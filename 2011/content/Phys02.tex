\documentclass[../ap_2011.tex]{subfiles}

\graphicspath{{./image/}}

\begin{document}

\chapter{統計力学: グランドカノニカル分布}
\(\beta=1/k_BT\)とする。
\section{}
スピンの状態を\(\sigma\)で指定する。
\(l_x,l_y,l_z,\sigma\)で指定される状態の分配関数は
\begin{equation}\begin{aligned}[b]
    \Xi_1(l_x,l_y,l_z,\sigma) = 1+\exp[-\beta\qty(\frac{2\pi^2\hbar^2}{mL^2}(l_x^2+l_y^2+l_z^2)-\mu)]
\end{aligned}\end{equation}
今、\(\sigma\)の状態には注目しないので、\(\sigma\)をトレースアウトすると
\begin{equation}\begin{aligned}[b]
    \Xi_1(l_x,l_y,l_z)&=\sum_{\sigma=\pm1/2}\Xi_1(l_x,l_y,l_z,\sigma)\\
    &=1+2\exp[-\beta\qty(\frac{2\pi^2\hbar^2}{mL^2}(l_x^2+l_y^2+l_z^2)-\mu)]+\exp[-2\beta\qty(\frac{2\pi^2\hbar^2}{mL^2}(l_x^2+l_y^2+l_z^2)-\mu)]
\end{aligned}\end{equation}
となる(これ自体は\(\sigma\)の自由度を考えずにはじめから、電子がないとき、スピンup/downが1つあるとき、電子が2つあるときの和をとっても得られる)。
また、\(\Xi\)の表式については\(\Xi_1\)をすべての状態についてそれぞれ考え、それらの積を取ればとよいので
\begin{equation}\begin{aligned}[b]
    \Xi = \prod_{l_x,l_y,l_z}\Xi_1(l_x,l_y,l_z)
\end{aligned}\end{equation}
のように書ける。

\section{}
\begin{equation}\begin{aligned}[b]
    \ev{n_k}&=\frac{1}{\Xi_1(k)}\sum_{n_k=0,1,1,2}n_ke^{-\beta(\varepsilon_k-\mu)n_k}
    =\frac{1}{\beta\Xi_1(k)}\pdv{\mu}\sum_{n_k=0,1,1,2}e^{-\beta(\varepsilon_k-\mu)n_k}\\
    &=\frac{\partial}{\beta\partial\mu}\ln\Xi_1(k)
    =\frac{\partial}{\beta\partial\mu}\ln\qty(1+2e^{-\beta(\varepsilon_k-\mu)}+e^{-2\beta(\varepsilon_k-\mu)})\\
    &=\frac{2e^{-\beta(\varepsilon_k-\mu)}+2e^{-2\beta(\varepsilon_k-\mu)}}{1+2e^{-\beta(\varepsilon_k-\mu)}+e^{-2\beta(\varepsilon_k-\mu)}}
    =\frac{2e^{-\beta(\varepsilon_k-\mu)}(1+e^{-\beta(\varepsilon_k-\mu)})}{(1+e^{-\beta(\varepsilon_k-\mu)})^2}\\
    &=\frac{2}{1+e^{\beta(\varepsilon_k-\mu)}}
\end{aligned}\end{equation}
のようにフェルミ分布関数がスピン自由度も込みで得られる。

\section{}
\begin{equation}\begin{aligned}[b]
    J &= -k_BT\ln\Xi = -k_BT\sum_{l_x,l_y,l_z}\ln\qty(1+2e^{-\beta(\varepsilon_k-\mu)}+e^{-2\beta(\varepsilon_k-\mu)})\\
    &= -k_BTL^3\int d\varepsilon\, D(\varepsilon)\ln\qty(1+2e^{-\beta(\varepsilon-\mu)}+e^{-2\beta(\varepsilon-\mu)})
    =-2k_BTL^3\int d\varepsilon\, D(\varepsilon)\ln\qty(1+e^{-\beta(\varepsilon-\mu)})
\end{aligned}\end{equation}


\section{}
\(E(k)\)の期待値は\(\beta\)と\(D(\varepsilon)\)を用いて表すと
\begin{equation}\begin{aligned}[b]
    \ev{E} &= \frac{1}{L^3\Xi}\sum_k\sum_{n_k}n_kE(k)e^{-\beta E(k)n_k}
    =\frac{1}{L^3\Xi}\sum_k\sum_{n_k}\qty(-\pdv{\beta}E(k)e^{-\beta E(k)n_k})\\
    &= -\frac{1}{L^3}\pdv{\beta}\ln\Xi
    = -\pdv{\beta}\int d\varepsilon\, D(\varepsilon)\ln\qty(1+e^{-\beta(\varepsilon-\mu)})
    =2\int d\varepsilon\, D(\varepsilon)\frac{E}{1+e^{\beta E}}
\end{aligned}\end{equation}
と書ける。


\section{}
状態密度\(D(\varepsilon)\)の表式は
\begin{equation}\begin{aligned}[b]
    \int d\varepsilon\,D(\varepsilon)
    &:=\frac{1}{(2\pi)^3}\int dk\,4\pi k^2
    =\int d\varepsilon\, \dv{k}{\varepsilon}\frac{k^2}{2\pi^2}
    =\int d\varepsilon\, \frac{1}{2\pi^2}\frac{mk}{\hbar^2}
    =\int d\varepsilon\, \frac{1}{4\pi^2}\qty(\frac{2m}{\hbar^2})^{3/2}\sqrt{\varepsilon}
\end{aligned}\end{equation}
より
\begin{equation}\begin{aligned}[b]
    D(\varepsilon)=\frac{1}{4\pi^2}\qty(\frac{2m}{\hbar^2})^{3/2}\sqrt{\varepsilon}
\end{aligned}\end{equation}
である。

\section{}
\(E(k)\)の絶対零度での期待値である\(E_g\)は
\begin{equation}\begin{aligned}[b]
    E_g &= 2\int_0^\mu d\varepsilon\,\frac{1}{4\pi^2}\qty(\frac{2m}{\hbar^2})^{3/2}\sqrt{\varepsilon}(\varepsilon-\mu)
    =\frac{1}{2\pi^2}\qty(\frac{2m}{\hbar^2})^{3/2}\qty[\frac{2}{5}\mu^{5/2}-\frac{2}{3}\mu^{5/2}]
    = -\frac{2}{15\pi^2}\qty(\frac{2m}{\hbar^2})^{3/2}\mu^{5/2}
\end{aligned}\end{equation}
である。
\footnote{見慣れない表式なのでこの式を観察してみる。
    エネルギーが負なのは、この問題でのエネルギーの基準が化学ポテンシャルであり、電子はそれより低いエネルギーに詰まっているからとわかる。
    また、これを物性の本で見るような表式にする。
    粒子数は状態蜜度にスピンの自由度を入れないnotationであるので\(N=\frac{2}{3}2D(\varepsilon_F)\varepsilon_F\)となる。
    これより
    \begin{equation*}\begin{aligned}[b]
        E_g =\frac{1}{2\pi^2}\qty(\frac{2m}{\hbar^2})^{3/2}\qty[\frac{2}{5}\varepsilon_F^{5/2}-\frac{2}{3}\varepsilon_F^{5/2}]\
        =\frac{2}{3}\underset{=2D(\varepsilon_F)}{\underline{\frac{1}{2\pi^2}\qty(\frac{2m}{\hbar^2})^{3/2}\sqrt{\varepsilon_F}}}\times\qty(\frac{3}{5}-1)\varepsilon_F
        =\frac{3}{5}N\varepsilon_F-N\varepsilon_F = -\frac{2}{5}N\varepsilon_F
    \end{aligned}\end{equation*}
    となる。
    フェルミ準位まで詰まった電子のエネルギーが\(\frac{3}{5}N\varepsilon_F\)でそこから、
    エネルギーの基準点であるフェルミエネルギーの分を除いた量とわかる。}

\section{}
粒子数密度は
\begin{equation}\begin{aligned}[b]
    \ev{\rho}&=\frac{1}{L^3\Xi}\sum_k\sum_{n_k}n_ke^{-\beta (\varepsilon_k-\mu)n_k}
    =\frac{1}{L^3\Xi}\sum_k\sum_{n_k}\frac{\partial}{\beta\partial\mu}e^{-\beta (\varepsilon_k-\mu)n_k}
    =\frac{1}{L^3}\frac{\partial}{\beta\mu}\ln\Xi\\
    &=2\int d\varepsilon\, D(\varepsilon)\ln(1+e^{-\beta(\varepsilon-\mu)})
    =2\int d\varepsilon\,\frac{D(\varepsilon)}{1+e^{\beta(\varepsilon-\mu)}}
\end{aligned}\end{equation}
と表せる。\(\rho=\partial_\mu J/L^3\)でもよい。

% これより密度感受率は
% \begin{equation}\begin{aligned}[b]
%     \kappa = \pdv{\mu}2\int d\varepsilon\,\frac{D(\varepsilon)}{1+e^{\beta(\varepsilon-\mu)}}
%     =2\int d\varepsilon\,\frac{\beta e^{\beta(\varepsilon-\mu)}}{1+e^{\beta(\varepsilon-\mu)}}D(\varepsilon)
%     =\beta\int d\varepsilon\,\frac{D(\varepsilon)}{1+\cosh(\beta(\varepsilon-\mu))}
% \end{aligned}\end{equation}
% と表せる。

絶対零度においては
\(\mu<0\)のとき\(\ev{\rho}=0\)より\(\kappa=0\)である。
\(\mu>0\)のとき、
\begin{equation}\begin{aligned}[b]
    \ev{\rho} = 2\int_0^\mu d\varepsilon\, \frac{1}{4\pi}\qty(\frac{2m}{\hbar^2})^{3/2}\sqrt{\varepsilon}
    =\frac{1}{3\pi^2}\qty(\frac{2m\mu}{\hbar^2})^{3/2}
\end{aligned}\end{equation}
より
\begin{equation}\begin{aligned}[b]
    \kappa = \frac{1}{2\pi^2}\qty(\frac{2m}{\hbar^2})^{3/2}=2D(\mu)
\end{aligned}\end{equation}
となり、スピンも考慮した状態密度と同じものになっている。

この結果はとなる理由は、
化学ポテンシャルを単位エネルギーずらしたときに増える電子の数を言い換えると
状態密度に単位エネルギーを掛けて得られる状態数そのものであるから。

つまり、化学ポテンシャルを変化させたとき、
フェルミ面付近にどれほど電子が増えるかというのが密度感受率と読むことができる。

\section{}
密度のゆらぎは
\begin{equation}\begin{aligned}[b]
    \ev{\rho^2}-\ev{\rho}^2
    &=\frac{1}{L^6\Xi}\sum_k\sum_{n_k}n_k^2e^{-\beta(\varepsilon_k-\mu)n_k}-\qty(\frac{1}{L^3\Xi}\sum_k\sum_{n_k}n_ke^{-\beta(\varepsilon_k-\mu)n_k})^2
    =\frac{1}{L^6\Xi}\frac{\partial^2\Xi}{\beta\partial\mu^2}-\qty(\frac{1}{L^3\Xi}\frac{\partial\Xi}{\beta\partial\mu})^2\\
    &=\frac{\partial}{L^3\beta\partial\mu}\qty(\frac{1}{\Xi}\frac{\partial \Xi}{L^3\beta\partial\mu})
    =\frac{\partial\ev{\rho}}{L^3\beta\partial\mu} = k_BT\frac{\kappa}{L^3} (= k_BT \frac{2D(\mu)}{L^3})
\end{aligned}\end{equation}
となる。

前問より密度感受律はフェルミ面付近の電子数の変化を表す量であるので、
フェルミ面付近の粒子数のゆらぎは\(k_BT\)に比例するという結果になる。
これはフェルミ分布関数の値の変化する領域が\(k_BT\)程度であるという[2]の結果とも一致する。


\section*{感想}
グランドカノニカル分布からフェルミ分布を導出する際に、
はじめからスピンの自由度を考慮してやるのは初めてやった。
これがよくやってる計算と同じ結果になったのは感動した。
よくあるスピン自由度を後から考えて状態密度を2倍するという処方だと、
エニオンのような分数統計のときには不安になるが、
この問題の手順だと安心して考えれる。

\end{document}
