\documentclass[../ap_2011.tex]{subfiles}

\graphicspath{{./image/}}

\begin{document}
\setcounter{chapter}{2}
\chapter{電磁気学: ゼーマン分裂の古典論}
\section{}
\begin{equation}\begin{aligned}[b]
    m\ddot{\vb*{r}} = -k\vb*{r}+q\dot{\vb*{r}}\times \vb*{B}\\
    \begin{cases}
        m\ddot{x} &= -kx +qB\dot{y}\\
        m\ddot{y} &= -ky -qB\dot{x}\\
        m\ddot{z} &= -kz
    \end{cases}
\end{aligned}\end{equation}
これより、\(z\)成分は単振動することがわかるので固有角振動数は
\begin{equation}\begin{aligned}[b]
    \omega_0 = \sqrt{\frac{k}{m}}
\end{aligned}\end{equation}

\section{}
前問の運動方程式に問で与えられた解の形を代入すると
\begin{equation}\begin{aligned}[b]
    &\begin{cases}
        -m\Omega^2 a\cos\Omega t &= -ka\cos\Omega t + qB\Omega b\cos\Omega t\\
        -m\Omega^2 b\sin\Omega t &= -kb\sin\Omega t + qB\Omega a\cos\Omega t
    \end{cases}\\
    &\begin{pmatrix}
        \omega_0^2-\Omega^2 & -\frac{qB}{m}\Omega\\
        -\frac{qB}{m}\Omega & \omega_0^2-\Omega^2
    \end{pmatrix}\begin{pmatrix}
        a\\b
    \end{pmatrix}=0
\end{aligned}\end{equation}
となる。この連立方程式が非自明な解を持つ条件より
\begin{equation}\begin{aligned}[b]
    0 &= (\omega_0^2-\Omega^2)^2-\qty(\frac{qB}{m}\Omega)\\
    &= \qty(\Omega^2-\frac{qB}{m}\Omega -\omega_0^2)\qty(\Omega^2+\frac{qB}{m}\Omega -\omega_0^2)\\
    \Omega &= \omega_0\sqrt{1+\qty(\frac{qB}{2m\omega_0})^2}\pm\frac{qB}{2m},\quad -\omega_0\sqrt{1+\qty(\frac{qB}{2m\omega_0})^2}\pm\frac{qB}{2m}
\end{aligned}\end{equation}
このうち振動数が正のものが求める解なので
\begin{equation}\begin{aligned}[b]
    \Omega = \omega_0\sqrt{1+\qty(\frac{qB}{2m\omega_0})^2}\pm\frac{qB}{2m}
\end{aligned}\end{equation}

\section{}
\begin{equation}\begin{aligned}[b]
    \Omega \simeq \omega_0\qty{1+\frac{1}{2}\qty(\frac{qB}{2m})^2}\pm\frac{qB}{2m}
    =\omega_0 \pm\frac{qB}{2m} +\frac{\omega_0}{2}\qty(\frac{qB}{2m\omega_0})^2
\end{aligned}\end{equation}

\section{}
磁場を掛ける前には角振動数\(\omega_0\)で右回りと左回りに荷電粒子が回る状態は縮退していたが、
磁場を掛けるとローレンツ力によって加速する回り方と減速する回り方になるため、
右周りと左回りで回転速度が変わるため。
(束縛された荷電粒子の軌道運動が磁場をかけたことによりゼーマン分裂が生じたから。)

\section{}
\(z\)方向は減衰するため、定常解には影響しない。
\(x,y\)方向の運動方程式は
\begin{equation}\begin{aligned}[b]
    &\begin{cases}
        m\ddot{x} &= -kx -m\gamma\dot{x} +qB\dot{y}+qE\cos\omega t\\
        m\ddot{y} &= -ky -m\gamma\dot{y} -qB\dot{x}
    \end{cases}\\
    &\begin{cases}
        \ddot{x} &= -\omega_0^2x -\gamma\dot{x} +\frac{qB}{m}\dot{y}+\frac{qE}{m}\cos\omega t\\
        \ddot{y} &= -\omega_0^2 y -\gamma\dot{y} -\frac{qB}{m}\dot{x}
    \end{cases}\\
    &\begin{cases}
        \dv[2]{t}(x+iy) &= -\omega_0^2(x+iy) -\qty(\gamma+i\frac{qB}{m})\dv{t}(x+iy)+\frac{qE}{m}\cos\omega t\\
        \dv[2]{t}(x-iy) &= -\omega_0^2(x-iy) -\qty(\gamma-i\frac{qB}{m})\dv{t}(x-iy)+\frac{qE}{m}\cos\omega t
    \end{cases}
\end{aligned}\end{equation}
のようにまとめることができる。ここで、\(u_\pm=x\pm iy\)のようにおくと運動方程式は
\begin{equation}\begin{aligned}[b]
    \ddot{u}_\pm +\omega_0^2u_\pm +\qty(\gamma\pm i\frac{qB}{m})\dot{u}_\pm =\frac{qE}{m}\cos\omega t
\end{aligned}\end{equation}
のようにまとめれる。

これの定常解の形を
\begin{equation}\begin{aligned}[b]
    u_{\pm} = A_{\pm}e^{i\omega t}+B_{\pm}e^{-i\omega t}
\end{aligned}\end{equation}
のように置く。
すると調べる量は
\begin{equation}\begin{aligned}[b]
    \overline{x(t)^2+y(t)^2}
    =\overline{u_+(t)u_-(t)}
    =\overline{A_+B_- +A_-B_+ + A_+A_-e^{2i\omega t} +B_+B_- e^{-2i\omega t}}
    =A_+B_- +A_-B_+
\end{aligned}\end{equation}
とわかる。

運動方程式に代入して、\(e^{i\omega t}\)と\(e^{-i\omega t}\)の係数を見ると
\begin{equation}\begin{aligned}[b]
    \qty(\omega_0^2-\omega^2\mp\omega\frac{qB}{m}+i\omega \gamma)A_\pm &= \frac{qE}{2m}\\
    \qty(\omega_0^2-\omega^2\pm\omega\frac{qB}{m}-i\omega \gamma)B_\pm &= \frac{qE}{2m}
\end{aligned}\end{equation}
これより
\begin{equation}\begin{aligned}[b]
    A_+B_-
    &= \qty(\frac{qE}{2m})^2\frac{1}{\omega_0^2-\omega^2-\omega\frac{qB}{m}+i\omega\gamma}\frac{1}{\omega_0^2-\omega^2-\omega\frac{qB}{m}-i\omega\gamma}\\
    &= \qty(\frac{qE}{2m})^2\frac{1}{\qty(\omega_0^2-\omega^2-\omega\frac{qB}{m})^2+\omega^2\gamma^2}\\
    A_-B_+
    &= \qty(\frac{qE}{2m})^2\frac{1}{\omega_0^2-\omega^2+\omega\frac{qB}{m}+i\omega\gamma}\frac{1}{\omega_0^2-\omega^2+\omega\frac{qB}{m}-i\omega\gamma}\\
    &= \qty(\frac{qE}{2m})^2\frac{1}{\qty(\omega_0^2-\omega^2+\omega\frac{qB}{m})^2+\omega^2\gamma^2}
\end{aligned}\end{equation}
よって
\begin{equation}\begin{aligned}[b]
    \overline{x(t)^2+y(t)^2}
    &=\frac{1}{2}\qty(\frac{qE}{2m})^2\qty[\frac{1}{\qty(\omega_0^2-\omega^2-\omega\frac{qB}{m})^2+\omega^2\gamma^2}
    +\frac{1}{\qty(\omega_0^2-\omega^2+\omega\frac{qB}{m})^2+\omega^2\gamma^2}]
\end{aligned}\end{equation}
この量は光を入れたときのゼーマン分裂があるときの吸収スペクトルに対応する。

\section{}
\subsection*{\(\gamma\ll qB/m \ll \omega_0\)の場合}
\(\gamma\ll qB/m\)というのは、基本\(\gamma\)を考えなくてもよいが、
\((\omega_0^2-\omega^2\mp\omega\frac{qB}{m})^2\)が\(0\)になるようなときには考慮しなければならないことを意味する。

そのような振動数は\(\omega = \Omega= \omega_0 \mp qB/2m\)であると設問[2]で求めた。
そのとき\(\omega_0\gg qB/2m\)より
\begin{equation}\begin{aligned}[b]
    \omega_0^2\mp\omega\frac{qB}{m}-\omega^2
    \simeq\qty(\omega_0\mp\frac{qB}{2m}+\omega)\qty(\omega_0\mp\frac{qB}{2m}-\omega)
    \simeq 2\qty(\omega_0\mp\frac{qB}{2m})\qty(\omega_0\mp\frac{qB}{2m}-\omega)
    \simeq 2\omega_0\qty(\omega_0\mp\frac{qB}{2m}-\omega)
\end{aligned}\end{equation}
\begin{equation}\begin{aligned}[b]
    \omega\gamma \simeq \qty(\omega_0\mp\frac{qB}{2m}) \gamma\simeq \omega_0 \gamma
\end{aligned}\end{equation}
よって
\begin{equation}\begin{aligned}[b]
    \overline{x(t)^2+y(t)^2}
    &=\frac{1}{2}\qty(\frac{qE}{2m})^2\qty[\frac{1}{\qty(\omega_0^2-\omega^2-\omega\frac{qB}{m})^2+\omega^2\gamma^2}
    +\frac{1}{\qty(\omega_0^2-\omega^2+\omega\frac{qB}{m})^2+\omega^2\gamma^2}]\\
    &\simeq\frac{1}{2}\qty(\frac{qE}{2m})^2\qty[\frac{1}{4\omega_0^2\qty(\omega_0-\frac{qB}{2m}-\omega)^2+\omega_0^2\gamma^2}
    +\frac{1}{4\omega_0^2\qty(\omega_0+\frac{qB}{2m}-\omega)^2+\omega_0^2\gamma^2}]\\
    &\simeq\frac{1}{2}\qty(\frac{qE}{4m\omega_0})^2\qty[\frac{1}{\qty(\omega_0-\frac{qB}{2m}-\omega)^2+\gamma^2/4}
    +\frac{1}{\qty(\omega_0+\frac{qB}{2m}-\omega)^2+\gamma^2/4}]
\end{aligned}\end{equation}

\subsection*{\(qB/m\ll\gamma \ll  \omega_0\)の場合}
1つ前の場合では\(\gamma\ll qB/2m\)という条件を使ってなかった。
今の場合ではこれを考慮すればよく、それには\(qB/m\to0\)とすればよいので、
\begin{equation}\begin{aligned}[b]
    \overline{x(t)^2+y(t)^2}
    &\simeq\frac{1}{2}\qty(\frac{qE}{4m\omega_0})^2\qty[\frac{1}{\qty(\omega_0-\frac{qB}{2m}-\omega)^2+\gamma^2/4}
    +\frac{1}{\qty(\omega_0+\frac{qB}{2m}-\omega)^2+\gamma^2/4}]\\
    &= \qty(\frac{qE}{4m\omega_0})^2\frac{1}{\qty(\omega_0-\omega)^2+\gamma^2/4}
\end{aligned}\end{equation}
となる。
つまり、\(\gamma\)によるぼやけが強すぎてゼーマン分裂が見えず、単一のローレンツピークとなったと解釈できる。
(\url{https://www.desmos.com/calculator/qboaxcxtuc})

\section*{感想}
ゼーマン分裂は量子論でやることが多いけど、古典論でも同じ結果が得られるのは感動しますよね。
ただ、後半の計算量が多いように感じた。何か早くやる方法はないのだろうか?


\end{document}
