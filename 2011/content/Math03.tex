\documentclass[../ap_2011.tex]{subfiles}

\graphicspath{{./image/}}

\begin{document}

\setcounter{chapter}{2}
\chapter{複素解析}
\section{}
極は\(z=i\)であり、そこでの留数は
\begin{equation}\begin{aligned}[b]
    \Res[f,z=i]=\lim_{z\to i}(z-i)f(z)=\lim_{z=i}\frac{\log(z+i)}{z+i}=\frac{\ln2+i\pi/2}{2i}=\frac{\pi}{4}-i\frac{\ln2}{2}
\end{aligned}\end{equation}
となる。
\section{}
留数定理より
\begin{equation}\begin{aligned}[b]
    \int_C f(z)dz = 2\pi i\Res[f,z=i]=\pi\ln2+i\frac{\pi^2}{2}
\end{aligned}\end{equation}

\section{}
三角不等式を使うと
\begin{equation}\begin{aligned}[b]
    \abs{\int_{C_1}\frac{\log(z+i)}{z^2+i}dz}
    &=\abs{\int_{C_1}frac{\log(z+i)}{z^2+i}dz}\\
    &\leq\int_{C_1}\frac{\abs{\log(z+i)}}{\abs{z^2+i}}\abs{dz}
\end{aligned}\end{equation}
耐えられた不等式と
\begin{equation}\begin{aligned}[b]
    \abs{z^2+i}=\sqrt{1+R^4+2R^2\sin2\theta}>\sqrt{R^4-2R^2+1}=R^2-1
\end{aligned}\end{equation}
を使うと
\begin{equation}\begin{aligned}[b]
    \abs{\int_{C_1}\frac{\log(z+i)}{z^2+i}dz}
    <\int_{0}^\pi\frac{\ln(R+1)+3\pi/2}{R^2-1}\abs{dz}
    =\pi R\frac{\ln(R+1)+3\pi/2}{R^2-1}
\end{aligned}\end{equation}

\section{}
\(g(z)\)の極は\(z=-i\)で留数は
\begin{equation}\begin{aligned}[b]
    \Res[g,z=-i]=\lim_{z\to -i}(z+i)g(z)=\lim_{z\to-i}\frac{\log(z-i)}{z-i}=\frac{\ln2-i\pi/2}{-2i}=\frac{\pi}{4}+i\frac{\ln2}{2}
\end{aligned}\end{equation}
これより留数定理を使うと
\begin{equation}\begin{aligned}[b]
    \int_\Gamma g(z)dz = 2\pi i\Res[g,z=-i]=-\pi\ln 2+i\frac{\pi^2}{2}
\end{aligned}\end{equation}

\end{document}
