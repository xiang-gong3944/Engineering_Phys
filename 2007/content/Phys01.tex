\documentclass[../../master.tex]{subfiles}

\graphicspath{{./image/}}

\begin{document}

\chapter{量子力学: 2準位系・相互作用表示・回転波近似}
\section{}
\begin{equation}\begin{aligned}[b]
    H_{aa} &= \bra{a}H\ket{b} = \hbar \omega_a\\
    H_{ab} &= \bra{a}H\ket{b} = -\mu\cos\nu t\\
    H_{ba} &= \bra{b}H\ket{b} = -\mu\cos\nu t\\
    H_{bb} &= \bra{b}H\ket{b} = \hbar \omega_b
\end{aligned}\end{equation}
\section{}
\(\sigma_x = \ket{a}\bra{b}+\ket{b}\bra{a}\)とおく。
シュレディンガー方程式より
\begin{equation}\begin{aligned}[b]
    i\hbar\pdv{t}\ket{\psi(t)} &= \qty(H_0-\sigma_x\mu\cos\nu t)\ket{\psi(t)}\\
    i\hbar\pdv{t}\qty(e^{-iH_0t/\hbar}\ket{\phi(t)}) &= \qty(H_0-\sigma_x\mu\cos\nu t)e^{-iH_0t/\hbar}\ket{\phi(t)}\\
    e^{-iH_0t/\hbar}i\hbar\pdv{t}\ket{\phi(t)} &= -\sigma_x\mu\cos\nu t e^{-iH_0t/\hbar}\ket{\phi(t)}\\
    i\hbar\pdv{t}\ket{\phi(t)} &= -\mu\cos\nu t e^{iH_0t/\hbar}\sigma_x e^{-iH_0t/\hbar}\ket{\phi(t)}
\end{aligned}\end{equation}
ここで、
\begin{equation}\begin{aligned}[b]
    e^{-iH_0t/\hbar}
    = e^{-i\omega_a t}\ket{a}\bra{a}+e^{-i\omega_b t}\ket{b}\bra{b}
    = e^{-i\omega_a t}\qty(\ket{a}\bra{a}+e^{-i\omega t}\ket{b}\bra{b})
\end{aligned}\end{equation}
と表せることを使うと
\begin{equation}\begin{aligned}[b]
    -\mu\cos\nu t e^{iH_0t/\hbar}\sigma_x e^{-iH_0t/\hbar}
    &= -\mu\cos\nu t \qty(\ket{a}\bra{a}+e^{i\omega t}\ket{b}\bra{b})(\ket{a}\bra{b}+\ket{b}\bra{a})
        \qty(\ket{a}\bra{a}+e^{-i\omega t}\ket{b}\bra{b})\\
    &= -\frac{\mu}{2}(e^{i\nu t}+e^{-i\nu t})\qty(e^{-i\omega t}\ket{a}\bra{b}+e^{-i\omega t}\ket{b}\bra{a})\\
    &= V(t)
\end{aligned}\end{equation}
とわかるので、
\begin{equation}\begin{aligned}[b]
    i\hbar\pdv{t}\ket{\phi(t)} = V(t)\ket{\phi(t)}
\end{aligned}\end{equation}

\section{}
\(V(t)=-\mu\sigma_x\)なので
\begin{equation}\begin{aligned}[b]
    -\frac{i}{\hbar}\int_{0}^{t}dt' V(t') &= -\frac{i\mu t}{\hbar}\sigma_x
\end{aligned}\end{equation}
となる。
よって
\begin{equation}\begin{aligned}[b]
    \ket{\phi(t)}
    &= \exp(-\frac{i\mu t}{\hbar}\sigma_x) \ket{\phi(0)}\\
    &= \qty(\cos\frac{\mu t}{\hbar}\sigma_x^2 - i\sin\frac{\mu t}{\hbar}\sigma_x)\ket{\phi(0)}\\
    &= \cos\frac{\mu t}{\hbar}\ket{a} - i\sin\frac{\mu t}{\hbar}\ket{b}
\end{aligned}\end{equation}

\section{}
\begin{equation}\begin{aligned}[b]
    \ket{\psi(t)} = e^{-iH_0t/\hbar}\ket{\phi(t)}=e^{-i\omega_a t}\cos\frac{\mu t}{\hbar}\ket{a} - ie^{-i\omega_b t}\sin\frac{\mu t}{\hbar}\ket{b}
\end{aligned}\end{equation}

\section*{感想}
J.J. Sakuraiでブラケット表記を始めて勉強したときにはなかなか行列に見えなくて頭を抱えたもんだなぁと懐かしくなった。


\end{document}
