\documentclass[../../master.tex]{subfiles}

\graphicspath{{./image/}}

\begin{document}

\chapter{物性: 金属・パウリ常磁性・ゾンマーフェルト展開}
略

\section*{感想}
金属まわりで出しやすい話はここら辺なのかな。
自分が印象に残ってるゾンマーフェルト展開の覚えた方は
\begin{equation*}\begin{aligned}[b]
    -\pdv{f}{\epsilon}
    = \delta(\epsilon-\epsilon_F) + \frac{\pi^2}{6}(k_BT)^2\delta''(\epsilon-\epsilon_F) +\mathcal{O}(k_BT^4)
\end{aligned}\end{equation*}
というようにみなすやり方。「金属絶縁体相転移」っていう本のなかで紹介されてた。


\end{document}
